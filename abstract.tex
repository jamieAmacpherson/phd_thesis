\section*{Abstract}
Allostery is the regulation of protein function from a remote non-active binding site. Ligand binding incites a rapid process whereby information travels significant biomolecular distances to enact energetic changes to the the protein's functional centre. Despite the demonstrated importance of protein allostery in maintaining cellular homeostasis, investigating the mechanisms over the molecular temporal and spatial scales remains a challenge.
%
%
\\\\
%
%
Fructose 1,6-bisphosphate (FBP) activates pyruvate kinase M2 (PKM2) through a well-described feed-forward process in glycolysis. Less understood, however, is how allosteric regulation of PKM2 occurs on a molecular level. Moreover, as several amino acids can compete for an additional binding site, it has thus far been unclear how concurrently bound ligands control PKM2 catalysis.
%
%
\\\\
%
%
This thesis shows that, in a panel of cancer cell lines, the intracellular concentration of FBP exceeds that which is required to fully saturate binding to PKM2. Moreover, PKM2 is exposed to a dynamic concentration range of amino acids including the activator L-serine (Ser) and the inhibitor L-phenylalanine (Phe). When FBP is constitutively bound, Ser and Phe competitively regulate the maximal velocity of the PKM2-catalysed reaction, independent from changes to the protein's oligomeric state. To investigate the molecular determinants of multi-ligand regulation, a novel computational method \textit{AlloHubMat} is developed and applied towards the analyses of molecular dynamics simulations of PKM2, identifying a number of 'allosteric hub' residues. A selection of the allosteric hub residues are mutated and an integrative approach is used to elucidate their impact on FBP- and Phe-mediated regulation of PKM2. 
%
%
\\\\
%
%
The findings herein demonstrate a role for residues involved in FBP-induced allostery in enabling the integration of allosteric input from Phe and reveal a mechanism that underlies the co-ordinate regulation of PKM2 activity by multiple allosteric ligands.


